\usepackage{xeCJK}
\setCJKmainfont{Microsoft JhengHei}

% Helper packages
\usepackage{xstring}    % for string comparison

% content alignment
\usepackage[newcommands]{ragged2e}
\usepackage{adjustbox}  % make table fit in \textwidth
\usepackage{tabularx}   % for making table whose column width is stretch
\usepackage{booktabs}   % prettify table
\usepackage{makecell}

\usepackage{enumitem}
\setlist{nosep}
% custom environment of enumerate/itemize
\newenvironment{myEnumerate}[1][]{\begin{enumerate}[topsep=-\parskip, #1]}
	{\end{enumerate}\leavevmode\ignorespacesafterend}
\newenvironment{myItemize}[1][]{\begin{itemize}[topsep=-\parskip, #1]}
	{\end{itemize}\leavevmode\ignorespacesafterend}
\newenvironment{steps}[1][]{\begin{enumerate}[topsep=\dimexpr-\parskip+5pt\relax, label={Step \arabic*:}, left=1.25em, itemsep=5pt, #1]}
	{\end{enumerate}\leavevmode\ignorespacesafterend}
	
% avoid top & bottom padding in "center" environment
\newenvironment{tightcenter}{%
  \setlength\topsep{0pt}%
  \setlength\parskip{0pt}%
  \par\centering}{\par\noindent\ignorespacesafterend}

% Create an empty line
\newcommand{\blankline}{\vskip \baselineskip}
\newcommand{\separator}{\par\noindent\rule{\textwidth}{0.4pt}}  % horizontal separator line

% layout setttings
% adjust \paragraph to the style I like
\renewcommand{\paragraph}[1]{\vskip \baselineskip \textbf{\S\ #1} \vskip 0.2\baselineskip}
\renewcommand{\subparagraph}[1]{\textbf{#1} \vskip -0.7\baselineskip}
% indent of paragraph
\setlength{\parskip}{\baselineskip}
\setlength{\parindent}{0pt}
\newcommand{\forceindent}{\leavevmode\parindent=3em\indent\parindent=0pt\relax}     % force indent locally while setting \parindent to 0 globally

\usepackage{geometry}
\newgeometry{top=2cm, bottom=2cm, left=2cm, right=2cm}

% Settings header and footer layout
\usepackage{fancyhdr}
\fancypagestyle{tocStyle}{      % page style for ToC
	\fancyhead{}           % clear header field by giving it no params
	\renewcommand{\headrulewidth}{0pt}     % clear the rule of header
	\fancyfoot[C]{\large\thepage}
	\pagenumbering{roman}     % set page number style to roman
	\setcounter{page}{1}
}

\fancypagestyle{contentStyle}{  % page style for content
	\setlength\headheight{15pt}
	\fancyhead{\fancycenter{\nouppercase\leftmark}{}{\nouppercase\rightmark}}
	\renewcommand{\headrulewidth}{0.4pt}   % switch back to default (0.4pt)
	\fancyfoot[C]{\large\thepage}
	\pagenumbering{arabic}     % turn on page number
	\setcounter{page}{1}
}

% images, captions, drawing
\usepackage{graphicx}   % for import images
\graphicspath{{image/}} % set image folder to "image/"

% font
\usepackage[light]{Chivo}
\renewcommand{\familydefault}{\sfdefault}       % change font family

% for text color
\usepackage[dvipsnames]{xcolor}
\usepackage{color}
\usepackage{soul}  % for text highlighting
\usepackage{cjkhl}

% In order to meet four requirements of URLs:
% (1) clickable link
% (2) line break
% (3) text color is blue
% (4) no need to escape special characters: %, _, #, and so on
% I've tried several ways and I note their results in the back (double quoted name are packages):
% 1. use \url in "url": satisfy (1) and (3)
% 2. use \url in "xurl" (loaded after "hyperref" package): satisfy (1) and (2). But if you do \textcolor{blue}{\url{some_link}}, you must to escape special characters, because of \textcolor
% 3. use \hyperref in "hyperref": satisfy (1) and (3)
% 4. load "xurl" before "hyperref", and use /url in "hyperref" (checked by changing the text color with \hypersetup of "hyperref"): satisfy (1), (2), (3) and (4).
% Why it works?
% Someone mentioned that "hyperref" will load "url" somewhere internally, and I know "xurl" will load "url" internally (according to its doc), so I think MAYBE if "xurl" loaded before "hyperref" and it can still change the behavior of \url in "hyperref". I really dont know what it exactly happened, but one thing we can know is, if there's two packages share the same name of command, we always use the command from the latest loaded package. In this case, that's "hyperref".
% P.S. Inspiration from the answer of this post: https://stackoverflow.com/questions/2625714/linebreak-in-url-with-bibtex-and-hyperref-package
\usepackage{xurl}           % enabling line break of \url. It should be loaded before "hyperref" package.
\usepackage{hyperref}       % creating hyperlink on table of content (ToC), and in-file reference
\hypersetup{
	colorlinks = true,
	linkcolor = {blue},
	urlcolor = {blue}
}

% customize reference
\newcommand*{\fullref}[1]{\hyperref[{#1}]{\ref*{#1} \nameref*{#1}}}

% Symbols in math or International System of Units (SI)
\usepackage{siunitx}            % for SI
\usepackage{newunicodechar}
\usepackage{mathtools}
\usepackage{amsmath}
\usepackage{amsthm}
\usepackage{amssymb}
\usepackage{cancel}             % add cancel line for math

% Define some commonly used mathematical symbols
\DeclareMathOperator{\Gen}{Gen}
\DeclareMathOperator{\Dec}{Dec}
\DeclareMathOperator{\Enc}{Enc}
\DeclareMathOperator{\dist}{dist}
\DeclareMathOperator{\K}{\mathcal{K}}
\DeclareMathOperator{\M}{\mathcal{M}}
\DeclareMathOperator{\C}{\mathcal{C}}
\DeclareMathOperator{\negl}{negl}   % for quiz in L4
% Since \DeclareMathOperator cannot accept arguments, so I use \newcommand instead.
\newcommand{\letter}[1]{\text{\textquotesingle{}{#1}\textquotesingle}}

% Define theorem for math
\newtheoremstyle{break}
  {\topsep}{\topsep}%
  {}{}%
  {\bfseries}{}%
  {\newline}{}%
\theoremstyle{break}
\newtheorem{theorem}{Theorem}
\newtheorem{definition}{Definition}

% Customized environment for proof
\makeatletter
\newenvironment{myProof}[1][\proofname]{\par
    \pushQED{\qed}%
    \normalfont \topsep6\p@\@plus6\p@\relax
    \trivlist
    \item[\hskip\labelsep
        {\itshape \bfseries{Proof}}%
        \IfStrEq{#1}{\proofname}{}{ (#1)} % 若 #1 為預設值,則不顯示括號
        ]\mbox{}\par\nobreak}
    {\popQED\endtrivlist\@endpefalse}
\makeatother



\definecolor{codeGray}{RGB}{227,230,232}
\newcommand{\codeblock}[1]{\texttt{\cjkhl{codeGray}{#1}}}

\newcommand{\entry}[2]{
	\begin{tabularx}{\linewidth}[t]{@{}lX@{}}
		\codeblock{#1} \qquad & \parbox[t]{\linewidth}{\raggedright #2}
	\end{tabularx}
}

\newcommand{\entryPair}[3]{
	\begin{tabularx}{\linewidth}{@{}lX@{}}
		\codeblock{#1} \: or \:  \codeblock{#2} \qquad & \parbox[t]{\linewidth}{\raggedright #3}
	\end{tabularx}
}

% line break for a long word
\newcommand{\breakandspace}{%
  \nobreak
  \hspace{0pt plus 1fil}% add a filling space
  \allowbreak
  \hspace{0pt plus -1fil}% remove the filling space
}

\ExplSyntaxOn
\NewDocumentCommand{\longcell}{m}
 {
  \text_map_inline:nn { #1 } { ##1\allowbreak }
  
 }
\ExplSyntaxOff

% ------- End -------

% Enabling long words to split/line break at arbitrary character.
% Note that seqsplit will eat all space in your string!
\usepackage{seqsplit}    
% I use the code just right above here, to replace \seqsplit
% This solution is adapted from egreg's answer from 
% https://tex.stackexchange.com/questions/207091/wrap-text-at-any-point-if-normal-hyphenation-not-possible
%\newcommand{\code}[1]{\texttt{\longcell{#1}}}
\newcommand{\code}[1]{\textcolor{purple}{\texttt{\longcell{#1}}}}

\newcommand{\myDescription}[2]{
	% using \myDescription in "itemize" environment while put another
	% "itemize" in arg2 will make the list icon (bullet) disappear.
	% The reason is unknown, but ChatGPT give a solution without any
	% side effect which is add \noindent in the beginning of command
	\noindent
	\begin{tabularx}{\linewidth}[t]{@{}l@{}X@{}}
		{#1:} & \parbox[t]{\linewidth}{\raggedright{}#2}
	\end{tabularx}
}



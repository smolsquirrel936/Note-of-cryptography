\section{L2: Perfect Secrecy}


\subsection{Encryption definition}

三個 space:
\begin{myItemize}
	\item \(\M\): message space
	\item \(\C\): ciphertext space
	\item \(\K\): key space
\end{myItemize}

三種動作:
\begin{myItemize}[itemsep=10pt]
	\item Gen (key generation): probabilistic algorithm。 \\
		\(\Gen(1^\lambda) \rightarrow k \in \K\), where \(\lambda\) is security parameter, or a symbol length (usually related to enc/dec execution time).
	\item Enc (encryption): probilistic algorithm。 \\
		For \(m \in \M\), \(\Enc_{k}(m) \rightarrow c \in \C\)
	\item Dec (decryption): deterministic algorithm。 \\
		For \(c \in \C\), \(\Dec_{k}(c) \coloneqq m \in \M\)
\end{myItemize}

注意上述使用 \(\rightarrow\) 表示 probabilistic algorithm;使用 \(\coloneqq\) 表示 deterministic algorithm。
Probabilistic algorithm 就是每次執行都有可能產生不同結果,而 deterministic algorthm 則代表每次執行必定產生出相同結果。

正確性 (Correctness) 定義: \\
\[\Pr[ \Dec_{k}(c) \coloneqq m : c \leftarrow \Enc_{k}(m), k \leftarrow \Gen(1^\lambda) ] = 1\]
即由正確的金鑰一定可以成功進行解密。 \\
對於某些系統,我們不一定會要求其機率是 1,可能會是接近 1 (即 \( \approx 1\))


\subsection{Notations}

Distribution over \(\K\):denoted as \(\dist(\K)\), which is defined by running \(\Gen\), and taking the output key。 \\
一個好的 key generation algorithm 應該要均勻地 (uniformly) 選擇 key(即選擇 key space 中的每個 key 的機率都是相等的)。因為如果我們有意地提高某些 key 的選擇機率,那麼攻擊者便可以藉由頻率分析知道我們的偏好,進而增加破解的機率。

\(K\):a random varaible, denoting the value of key generated by \(Gen\).

\(\Pr[K = k]\):for all \(k \in \K\), it denotes the probability that the key generated by \(\Gen\) is equal to \(k\).

上面三項皆可以套用至明文(\(\dist(\M)\)、\(M\)、\(\Pr[M = m]\))和密文(\(\dist(\C)\)、\(C\)、\(\Pr[C = c]\))。

當我們固定一個 encryption scheme \(\Pi = (\Gen, \Enc, \Dec)\) 且 dist over \(\M\),這就可以根據所給定的 \(k \in \K\) 和 \(m \in \M\),確定 \(\dist(\C)\)。


\subsection{Example of notations}


\paragraph{Example 1}

一個 adversary A 知道訊息是「attack today」的機率是 70\%、「not attack」的機率是 30\%,所以
\[\Pr[M=\mathrm{A.T.}] = 0.7, \quad \Pr[M=\mathrm{N.A.}] = 0.3\]

Random variables \(K\) 和 \(M\) 會假設沒有關係 (independent)。因為 \(\dist(\K)\) 由 \(\Gen\) 決定,而 \(\dist(M)\) 由我們想要加密的 context 決定。


\paragraph{Example 2 - Shift cipher}

\(K = \{0, 1, 2, \ldots, 25\}\) with \(\Pr[K = k] = \displaystyle{\frac{1}{26}}\) (aka uniformly distributed).

Let distribution of \(\M\)
\begin{equation*}
	\dist(\M) =
		\begin{cases}
			\Pr[M = \text{\textquotesingle{}a\textquotesingle}] = 0.7 \\
			\Pr[M = \text{\textquotesingle{}z\textquotesingle}] = 0.3
		\end{cases}
\end{equation*}

Then
\begin{align*}
	\Pr[C = \text{\textquotesingle{}b\textquotesingle}] &= 
			\Pr[M = \text{\textquotesingle{}a\textquotesingle} \wedge K = 1] +
			\Pr[M = \text{\textquotesingle{}z\textquotesingle} \wedge K = 2] \\
		&= \Pr[M = \text{\textquotesingle{}a\textquotesingle}] \cdot \Pr[K = 1] + 
			\Pr[M = \text{\textquotesingle{}z\textquotesingle}] \cdot \Pr[K = 2]
			\quad \text{(By independence)} \\
		&= 0.7 \cdot \frac{1}{26} + 0.3 \cdot \frac{1}{26} \\
		&= \frac{1}{26}
\end{align*}

Condition probability
\begin{align*}
	\Pr[M = \text{\textquotesingle{}a\textquotesingle} | C =  \text{\textquotesingle{}b\textquotesingle}] &=
	\frac{
		\Pr[C = \text{\textquotesingle{}b\textquotesingle} | M = \text{\textquotesingle{}a\textquotesingle}] \cdot \Pr[M = \text{\textquotesingle{}a\textquotesingle}]
	}
	{\Pr[C = \text{\textquotesingle{}b\textquotesingle}]} \\
	&= \frac{\frac{1}{26} \cdot 0.7}{\frac{1}{26}}
\end{align*}

where \, \(\Pr[C = \text{\textquotesingle{}b\textquotesingle} | M = \text{\textquotesingle{}a\textquotesingle}]\) \, means \, \(K = 1\), and\, \(\Pr[K = 1] = \frac{1}{26}\)

[Bayes' theorem]
\[\Pr[A|B] = \frac{\Pr[B|A] \cdot \Pr[A]}{\Pr[B]} \qquad \text{if} \quad \Pr[B] \neq 0\]


\subsection{Intuition for security}

Adversary 通常在收發兩端的中間進行竊聽 (eavesdrop)。 \\
Adversary 知道 \(\dist(\M)\) 和 encryption scheme \(\Pi = (\Gen, \Enc, \Dec)\),而不知道 key。

A scheme \(\Pi\) meets \textbf{perfect secrecy} means observation (usually from adversary) on ciphertext \(c\) should give no additional infomation. \\
意即密文 \(c\) 不能給攻擊者有更多的資訊可以更準確地進行猜測,也可以說 \(c\) 不會洩漏更多的資訊。


\subsection{Perfect secrecy}

\subparagraph{Formal definition of perfect secrecy (\textbf{Definition 1})}

An encrytion scheme \(\Pi = (\Gen, \Enc, \Dec)\) with message space \(\M\) is perfect secrecy if for every probability distribution over \(\M\), every message \(m \in \M\) and every chiphertext \(c \in \C\) for \(\Pr[C = c] > 0\)
\[\Pr[M = m | C = c] = \Pr[M = m]\]

簡單來說,就是在觀察 \(c\) 之後,所得知的 \(\dist(\M)\) 與在觀察 \(c\) 之前相等。 \\
若 \(c\) 洩漏了某些資訊,則上式中的等號 (=) 應該改成大於符號 (>)。


\subparagraph{Example: shift cipher}

這邊用和前面一樣的例子:
\begin{align*}
	\Pr[C = \text{\textquotesingle{}b\textquotesingle}] &= 
			\Pr[M = \text{\textquotesingle{}a\textquotesingle} \wedge K = 1] +
			\Pr[M = \text{\textquotesingle{}z\textquotesingle} \wedge K = 2] \\
		&= \Pr[M = \text{\textquotesingle{}a\textquotesingle}] \cdot \Pr[K = 1] + 
			\Pr[M = \text{\textquotesingle{}z\textquotesingle}] \cdot \Pr[K = 2]
			\quad \text{(By independence)} \\
		&= 0.7 \cdot \frac{1}{26} + 0.3 \cdot \frac{1}{26} \\
		&= \frac{1}{26}
\end{align*}

\(\Pr[C = \text{\textquotesingle{}b\textquotesingle}] = \displaystyle{\frac{1}{26}}\)

\begin{align*}
	\Pr[M = \text{\textquotesingle{}a\textquotesingle} | C =  \text{\textquotesingle{}b\textquotesingle}] &=
	\frac{
		\Pr[C = \text{\textquotesingle{}b\textquotesingle} | M = \text{\textquotesingle{}a\textquotesingle}] \cdot \Pr[M = \text{\textquotesingle{}a\textquotesingle}]
	}
	{\Pr[C = \text{\textquotesingle{}b\textquotesingle}]} \\
	&= \frac{\displaystyle{\frac{1}{26}} \cdot 0.7}{\displaystyle{\frac{1}{26}}} \\
	&= 0.7 \\
	&= \Pr[M = \text{\textquotesingle{}a\textquotesingle}]
\end{align*}

由此可知,shift cipher 是 prefect secrecy。


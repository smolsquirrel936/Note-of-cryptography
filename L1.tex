\section{L1}


\subsection{Merkle 的故事}

Merkle 在大學部修了一個課,然後要交一個 project。他在交這個作業的時候,提到了 Public Key Cryptography 的想法。當時的導師並不看好這個東西,所以 reject 了,最後他也退掉了這門課。之後他找到另一個很欣賞他的老師,覺得應該要「Publish it, win fame and fortune」,所以他將這篇文章那個投到了 CACM(Communications of the ACM)。第一次投期刊就因為「這個想法不是當今的主流想法」而被拒絕。在 Merkle 的某些堅持之下,過了快三年終於讓 CACM 接受了這篇文章。

這邊的故事及當時的論文,可以在 \url{https://ralphmerkle.com/1974/} 找到。

另外影片中的 link 有誤,應該改成 \url{https://ralphmerkle.com},不然你只會找到一間搞 CRM 和賣資料的公司。


\subsection{Conventions}

\begin{itemize}[itemsep=10pt]
	\item 離散且有限的時間 (discrete and finite world) \\
		\(\Rightarrow\) 因為我們正在討論 computer science
		
	\item Data v.s. Information
	
	\item Machine (function/algorithm)需要在 polynomial time 下執行 \\
		\(\Rightarrow\) 因為我們需要能在一定時間內看到結果,不想要等到天荒地老 \\
		\(\Rightarrow\) 不一定\textbf{強制}要求 polynomial time,但這堂課大部分會是這樣
		
	\item Alice and Bob:就是 sender 和 receiver,通常是 Alice 要傳訊息給 Bob \\
		\(\Rightarrow\) 還有其他角色,可以參見 Wikipedia: \\
		\url{https://en.wikipedia.org/wiki/Alice_and_Bob}
		
	\item 計算(computation):任何遵循 well-defined model(例如 algorithm、protocol)的 calculation。
	
	\item Efficiency \\
		Input size: \(|x| = n\) bits \\
		其他的就是拿 complexity 概念來作為 efficiency 的概念
		
	\item Crypto 像是信仰 (Faith)? \\
		密碼學不一定總是對的,但我們需要相信某些東西才能繼續在密碼學上前進 \\
		這些東西包含: \\ 
		\(\Rightarrow\) 某些數學問題很難被解決 \\
		\(\Rightarrow\) 某些假設無法被打破(通常指在 poly-time 底下) \\
		\(\Rightarrow\) 某些底層的密碼工具 (underlying crypto primitives) 是安全的 \\
		\(\Rightarrow\) P \(\neq\) NP \\
		\(\Rightarrow\) 亂數/隨機 (randomness),因為我們不知道真的亂數長什麼樣,所以無法驗證
\end{itemize}


\subsection{Overview}


\paragraph{什麼是密碼學?}

如果我們不在意安全,那麼我們不需要密碼學。 \\
(If do not care security, we won't need crypto.)

安全 (security) 可以由以下兩點來定義:
\begin{myItemize}
	\item 目的 (purposes):我們需要達到什麼效果
	\item 需求 (requirements):為了達到目的,我們需要達成哪些目標
\end{myItemize}

一些密碼學相關的內容:
\begin{itemize}
	\item 加密 (entryption)
	\item 數位簽章 (signature)
	\item 零知識 (zero knowledge)
	\item 安全計算 (secure computation)
\end{itemize}


\subsection{Notations}

\subparagraph{Private key encryption (or “secret key encryption”)}
就是對稱式加密,加密和解密皆使用同一個 key

\subparagraph{Public key encryption}
公鑰系統。一個公鑰會對應一個私鑰。公鑰會公開,私鑰不公開。 \\
若 Alice 要傳訊息給 Bob,則 Alice 會使用自己的公鑰加密,並且讓 Bob 使用「與 Alice 的公鑰相對應的」私鑰進行解密。

\subparagraph{Zero knowledge}
A 想向 B 證明某件事情,但不想透漏任何其他的額外資訊。 \\
Ex1:我想向你證明我有 100 萬,但不想真的放 100 萬現金在你眼前(以免被你搶走),所以我可以要求銀行開立證明來達到這個目的。
Ex2:我想向你證明我真的知道「威利在哪裡」。我可以用一張比原圖更大張的紙,並且在上面挖一個威利形狀的洞,以此來達到目的。


\subsection{Story of solving impossibility}

(這邊的例子經過一點點調整) \\
你的上司要求你解決一個問題 \(Q\),並且告知你如果無法解決問題就會被炒魷魚,並被另一個比你聰明的傢伙取代。你雖然不知道怎麼解決 \(Q\),但你知道另一個\textbf{相關的}知名問題 \(\widetilde{Q}\) (Q tilde) 在現今根本就沒人會解。最後你告訴你的上司,由於「現在根本沒人知道如何解 \(\widetilde{Q}\)」,所以「也沒人會解 \(Q\)」,因此這問題解不了,而另一個自稱聰明的傢伙其實是騙子。


重點就是 \\
If there's a good algorithm for \(Q\), then there exists a good one for another well-known problem \(\widetilde{Q}\). \\
這句話的逆否命題就是 \\
If there's no algorithm for \(\widetilde{Q}\), then there's no algorithm for \(Q\) either.

這背後的概念就是 reduction(就演算法的那個 reduction) 。


\subsection{Principle of modern crypto}

\subparagraph{Kerckhoff's principle}
「加密方法不能被要求是保密的,就算它落入敵人手中也不應該造成麻煩」 \\
意即,整套加密方法的安全性只仰賴金鑰的保密。

(原文: It should not require secrecy, and it should not be a problem if it falls into enemy hands.)

\subparagraph{Principle of modern crypto}
\begin{enumerate}[itemsep=10pt]
	\item Formal definition
		\begin{itemize}
			\item System framework (model):系統長什麼樣子
			\item Security definition:如何定義安全
		\end{itemize}
	\item Precise assumption \qquad \(\Pi'\)\\
		通常會是已知難題 \\
		從上一節的重點可以知道,我們通常會將加密法與某個已經被研究過的難題 (well-studied hardness) 做連結。若難題不是 well-studied,一來無法說服別人這個加密法安全,二來代表可能有人知道這個問題如何解決。
	\item Construction \qquad \(\Pi\)\\
		加密法的步驟是什麼
	\item Security proof \\
		基本上就是上一節的 reduction \\
		如果假象的攻擊者可以在 definition (即第一個要素)底下破解 \(\Pi\),那麼我可以構造另一個攻擊者,使其破解已知難題 \(\Pi'\)。 \\
		上面逆否命題的推論可以寫成:如果 \(\Pi'\) 是安全的(意即不被破解),那麼 \(\Pi\) 就是安全的。
\end{enumerate}

加密系統 = 產生 key (key generation) + 加密 (encryption) + 解密 (decryption)


\subsection{History of cryptography}

\paragraph{Shift cipher}

使用 private key encryption。 \\
Key 是每個字母需要做 shift 的次數。

Key generation:選擇一個  \(key \in \{0, 1, \ldots, 25\}\) \\
Encryption:將每個字母對應的數字 shift \(key\) 位 \\
Decryption:將每個字母對應的數字\textbf{反方向} shift \(key\) 位

破解:最多嘗試 26 次就可以找到答案


\paragraph{Substitution cipher}

使用 private key encryption。

Key generation:將每個字母逐一對應到另一個字母,以此這個 mapping 作為 key \\
Encryption:將明文中的字母按照 key 逐一對應過去 \\
Decryption:將密文中的字母按照 key 逐一對應回來

破解:字典攻擊(常用詞)+頻率分析(「E」在英文中出現的次數比較多)

加強:明文中不使用頻率較高的字母


\paragraph{Stronger cipher?}

Vigenère cipher:設定偏移量為字母在明文中所在的位置。

DES (first published in 1975, and standardized in 1977)

AES


\paragraph{History about PKC}

1974: Merkle proposed the notion \\
1976: Diffie-Hellman proposed the key exchange solution (Turing Awad 2015) \\
1977: Rivest-Shamir-Adleman proposed the first PKE (Turing Award 2002) \\
UK claimed their Government Communications Headquarters proposed such PKC idea before them.

Other impovements: ID-based encryption from Weil Pairing\\
使用了不同的 assumption,所以概念上較簡單,執行起來也較有效率
(關於 ID-based 的概念,之後如果有時間,可能會提到)

\section{L1}


\subsection{Merkle 的故事}

Merkle 在大學部修了一個課,然後要交一個 project。他在交這個作業的時候,提到了 Public Key Cryptography 的想法。當時的導師並不看好這個東西,所以 reject 了,最後他也退掉了這門課。之後他找到另一個很欣賞他的老師,覺得應該要「Publish it, win fame and fortune」,所以他將這篇文章那個投到了 CACM(Communications of the ACM)。第一次投期刊就因為「這個想法不是當今的主流想法」而被拒絕。在 Merkle 的某些堅持之下,過了快三年終於讓 CACM 接受了這篇文章。

這邊的故事及當時的論文,可以在 \url{https://ralphmerkle.com/1974/} 找到。

另外影片中的 link 有誤,應該改成 \url{https://ralphmerkle.com},不然你只會找到一間搞 CRM 和賣資料的公司。


\subsection{Conventions}

\begin{itemize}[itemsep=10pt]
	\item 離散且有限的時間 (Discrete and finite world) \\
		\(\Rightarrow\) 因為我們正在討論 computer science
		
	\item Data v.s. Informatition
	
	\item Machine (function/algorithm)需要在 polynomial time 下執行 \\
		\(\Rightarrow\) 因為我們需要能在一定時間內看到結果,不想要等到天荒地老 \\
		\(\Rightarrow\) 不一定\textbf{強制}要求 polynomial time,但這堂課大部分會是這樣
		
	\item Alice and Bob:就是 sender 和 receiver,通常是 Alice 要傳訊息給 Bob \\
		\(\Rightarrow\) 還有其他角色,可以參見 Wikipedia: \\
		\url{https://en.wikipedia.org/wiki/Alice_and_Bob}
		
	\item 計算(computation):任何遵循 well-defined model(例如 algorithm、protocol)的 calculation。
	
	\item Efficiency \\
		Input size: \(|x| = n\) bits \\
		其他的就是拿 complexity 概念來作為 efficiency 的概念
		
	\item Crypto 像是信仰 (Faith)? \\
		密碼學不一定總是對的,但我們需要相信某些東西才能繼續在密碼學上前進 \\
		這些東西包含: \\ 
		\(\Rightarrow\) 某些數學問題很難被解決 \\
		\(\Rightarrow\) 某些假設無法被打破(通常指在 poly-time 底下) \\
		\(\Rightarrow\) 某些底層的密碼工具 (underlying crypto primitives) 是安全的 \\
		\(\Rightarrow\) P \(\neq\) NP \\
		\(\Rightarrow\) 亂數/隨機 (randomness),因為我們不知道真的亂數長什麼樣,所以無法驗證
\end{itemize}












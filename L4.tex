\section{L4}


\subsection{Limitation of Perfect Secrecy}

\begin{theorem}[Limitation of perfect secrecy]
	If \(\Pi = (\Gen, \Enc, \Dec)\) is a perfectly secret encryption  scheme with message space \(\M\) and key space \(\K\), then
	\[|\M| \le |\K| \]
\end{theorem}

\begin{myProof}
	Suppose \(|\K| < |\M|\), \(\Pi\) cannot be perfectly secret.

	Consider the uniform \(\dist(\M)\) and fix \(c \in \C\), \(\Pr[C = c] = 0\).
	
	Let \(\M(c)\) be the set of possible message which contains all possible messages decrypted by \(c\). \\ That is,
	\[\M(c) \overset{\mathrm{def}}{=} \{m \mid m = \Dec_K(c) \text{ for some } k \in \K\}\]
	
	\(\Dec\) is deterministic function, so \(|\M(c)| \le |\K|\). \\
	(We know \(\Dec_k(c) \coloneq m\), and different values of \(k\) may map to the same \(m\). If all \(m\) are distinct for different \(k\), then equation holds; otherwise, \(|\M(c)| < |\K|\).)
	
	If \(|\K| < |\M|\) and \(\M(c) \le |\K|\), there exist some \(m' \in \M \) but \(m' \notin \M(c)\). \\
	\(\Rightarrow \Pr[M = m' \mid C = c] = 0 \neq \Pr[M = m']\), which is not perfect secrecy.
\end{myProof}

\subparagraph{Quiz}

We know that it's impossible to achieve pefect secrecy with shorter key size. So, what can we do or modify some factors to achieve shorter key? Any tradeoff (factor)?


\paragraph{Shannon's Theorem}

\begin{theorem}[Shannon's theorem]
	Let \(\Pi = (\Gen, \Enc, \Dec)\) be an encryption scheme with message space \(M\) for which \(|\M|=|\K|=|\C|\). \\
	The scheme is perfectly secret if and only if:
	\begin{enumerate}[topsep=-\parskip]
		\item Every key \(k \in \K\) is chosen with probability \(\displaystyle\frac{1}{|\K|}\) by \(\Gen\)
		\item For every \(m \in \M\) and every \(c \in \C\), there exists a unique key \(k \in \K\) such that \(\Enc_k(m)\).
	\end{enumerate}
\end{theorem}

\subparagraph{Quiz}

Design a tricky scheme \(\Pi\) that \(k \in \K\) is \textbf{NOT} uniformly chosen. Show \(\Pi\) is \textbf{NOT} perfectly secret by using Definition 1, 2 or 3. \\
(Hint: modify shift cipher or one-time pad)


\subsection{Private Key Encryption}


\paragraph{Computational Security}

Perfect secrecy 的缺點 (weakness):
\begin{myItemize}
	\item 只能用一次 (one time use)
	\item key 的長度一定要大於訊息的長度 (\(|\K| \geq |\M|\))
\end{myItemize}

Computational security 是從計算上保證安全的一種安全性。它不像 pefect secrecy 那樣地完美,但可以更靈活地建立 scheme(如減少 key 的長度)。

從 adversary 的觀點來看:
\begin{table}[h]
	\begin{tabular}{l@{\hspace{30pt}}l@{\hspace{30pt}}l} \toprule
		Adversary's power & time/space & success probability \\ \midrule
		Perfect secrecy & unbounded & = random guess \\
		Computational security & polynomial time & = random guess + small probability \\ \bottomrule
	\end{tabular}
\end{table}

目的:減少安全性,來換取更好的效率 (by weakening the security, to achieve better efficiency)。


\paragraph{Concrete Definition}

\begin{definition}[Concrete definition]
	A scheme \(\Pi\) is \((t,\epsilon)\)-secure if any adversary \(A\) running for time at most \(t\), succeeds in breaking \(\Pi\) with probability at most \(\epsilon\).
\end{definition}

Ex: \(t = 2^{10}\), \(\epsilon = \displaystyle\frac{1}{2^{100}}\)


\paragraph{Asymptotic Definition}

在這裡的 adversary \(A\) 的能力 (power) 是以漸進式術語來定義的 (asymptotic setting):
\begin{itemize}
	\item Efficient adversary:這種 adversary 會執行可以在 polynomial time 內跑完的演算法。這種演算法的執行時間是 \(p(n)\),其中 \(p\) 為多項式集合,而 \(n\) 為安全參數 (security parameter)。
	
	\item Small probability of success: 成功機率小於任何 polynomial 的倒數。也就是
	\[\Pr[\mathrm{success}] < \frac{1}{p(n)} \text{, where } p \text{ is arbitary polynomial}\]
\end{itemize}

PPT = Probabilistic Polynomial Time

\begin{definition}[Asymptotic definition]
	A scheme is secure if for any PPT adversary succeeds in breaking the scheme with at most \textbf{negligible} probability.
\end{definition}


\paragraph{Negligible Probability}

Negligible function 是漸進小於 (asymptotic smaller) 任何 polynomial function 的函數。

\begin{definition}
	A function \(f\) is negligible if \\
	for every positive polynomial \(p\), there exists a number \(N\) such that \(f(n) < \displaystyle\frac{1}{p(n)}\) where \(n > N\).
\end{definition}

Example: \\
Let \(g(x) = \displaystyle\frac{1}{2^x}\). \\
There exists \(N\) such that \(g(n) < \displaystyle\frac{1}{p(n)}\). \\
\begin{align*}
	&g(n) < \frac{1}{p(n)} \\
	&\Rightarrow \quad \frac{1}{2^n} < \frac{1}{n^k} \tag{k is positive constant} \\
	&\Rightarrow \quad 2^n > n^k \\
	&\Rightarrow \quad n > k \cdot log_2(n) \\
	&\Rightarrow \quad \frac{n}{log_2(n)} > k
\end{align*}

If \(n > k^2\), this inequality holds.

\subparagraph{Quiz}

Let \(\negl(x)\), \(\negl'(x)\) be negligible functions. \\
\begin{enumerate}
	\item A function \(f_1\), defined by \(f_1(x) = \negl(x) + \negl'(x)\)
	\item A function \(f_2\), defined by \(f_2(x) = p(x) \cdot \negl(x)\), where \(p(x)\) is positive polynomial.
\end{enumerate}

Are \(f_1\) and \(f_2\) are still negligible functions? \textbf{Yes}

\subparagraph{Summary}

任何關於 computational security 的 security definition 都由下列組成:
\begin{myEnumerate}
	\item 破解 scheme 的定義(也就是怎麼樣才叫 scheme 被破解了)
	\item 關於 adversary 的能力
\end{myEnumerate}

我們通常將 adversary 塑造 (model) 成有效率(有計算能力)的演算法,且只考慮 adversary 可以在 polynomial time 之內執行的 probabilistic stratigies。

\begin{definition}
	A scheme is secure if for every PPT adversary \(A\) carrying out an attack of some formally specified attack type, and the probability that \(A\) succeeds is negligible.
\end{definition}


\paragraph{Private Key Encryption}

\begin{definition}[Private key encryption]
	A private key encryption is a tuple of PPT algorithm \((\Gen, \Enc, \Dec)\) \\
	\begin{myItemize}
		\item Key generation: \(\Gen(1^k) \rightarrow k\). 這裡 \(n\) 的意義是 \(|\K| \geq n\) 或 \(|\K| = \mathrm{poly}(n)\)。
		
		\item Encryption: \(\Enc_k(m) \rightarrow c\), where key \(k\) and \(m \in \{0, 1\}*\) are inputs. 若 \(m \in \{0, 1\}^{l(n)}\),我們會稱這個等式為 fixed-length private key encryption with message length \(l(n)\)。
		
		\item Decryption: \(\Dec_k(c) \coloneq m\). If \(c\) cannot be decrypted, then outupt \(\bot\) (error). 
	\end{myItemize}
\end{definition}


\subparagraph{Basic definition of security}

Eavesdropper:adversary 的策略或能力

這裡和之前的 \(PrivK_{A, \Pi}^{eav}\) 大致一樣,參見 \fullref{PrivK}。

差異:
\begin{myItemize}
	\item Perfect secrecy:沒有 security parameter,因為不在意 adversary 有多少的能力
		\[\Pr[PrivK_{A, \Pi}^{eav} = 1] = \frac{1}{2}\]
	\item Computational security:有 security parameter \(n\)
		\[\Pr[PrivK_{A, \Pi}^{eav} = 1] \leq \frac{1}{2} + \negl(n)\]
\end{myItemize}

